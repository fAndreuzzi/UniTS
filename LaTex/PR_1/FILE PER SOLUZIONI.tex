\documentclass[11pt,reqno]{amsart}
\usepackage{geometry}                % See geometry.pdf to learn the layout options. There are lots.
\geometry{letterpaper}                   % ... or a4paper or a5paper or ...
%\geometry{landscape}                % Activate for for rotated page geometry
%\usepackage[parfill]{parskip}    % Activate to begin paragraphs with an empty line rather than an indent
\usepackage{graphicx}
\usepackage{lipsum}
\usepackage{amsmath}
\usepackage{amssymb}
\usepackage[makeroom]{cancel}
\DeclareGraphicsRule{.tif}{png}{.png}{`convert #1 `dirname #1`/`basename #1 .tif`.png}

%\date{}                                           % Activate to display a given date or no date

\title{Prova Riservata 1 -  pagine 1-11
}
\author{CONSEGNA: Gioved\`{\i} 6 DICEMBRE 2018 ore 12.00}
\begin{document}

\maketitle

\begin{figure}[ht!]
\centering
\includegraphics[width=30mm]{cunicio.jpg}
\end{figure}

{\bf Andreuzzi Francesco}

\bigskip
{\bf Numero di Matricola: IN0500630}

\bigskip
{\bf Laurea in Ingegneria Informatica (applicazioni informatiche)}

\bigskip

\bigskip

{\bf Approvazione questionario:}
\bigskip

\includegraphics[width=165mm]{curriculum.png}

\newpage

\centerline{\bf Soluzione Esercizio 1 }
\bigskip

\begin{gather*}
  \text{Applichiamo la disuguaglianza } A_m \geq G_m\\
  \frac{a_n + b_n}{2} \geq \sqrt{a_{n}b_n} \quad \implies \quad a_n \geq b_n \quad \forall \, n\\\\
  \bullet \quad a_n\text{ è}\textbf{ decrescente}\\
  1) \quad a_1 + b_1 < 2a_1 \implies \frac{a_1 + b_1}{2} < a_1 \quad \implies \quad a_2 < a_1\\
  n) \quad a_n + b_n \leq 2a_n \implies \frac{a_n + b_n}{2} \leq a_n \quad \implies \quad a_{n+1} \leq a_n\\\\
  \bullet \quad b_n\text{ è}\textbf{ crescente}\\
  1) \quad \sqrt{a_{1}b_1} > \sqrt{b_{1}b_1} = b_1 \quad \implies \quad b_2 > b_1\\
  n) \quad \sqrt{a_{n}b_n} > \sqrt{b_{n}b_n} = b_n \quad \implies \quad b_{n+1} > b_n\\\\
  \text{Quindi } a_n, b_n \text{ sono } \textbf{limitate}:\\
  a_1 \geq a_n \geq b_n \geq b_1 \geq 0 \quad \forall \, n\\\\
  a_n, b_n \text{ sono } \textbf{monotone} \text{ e } \textbf{limitate}\\
  \implies \exists \, L_a, L_b :\\
  \bullet \lim_{n \to \infty} a_n = L_a\\
  \bullet \lim_{n \to \infty} b_n = L_b\\\\
  \lim_{n \to \infty} a_n = L_a = \frac{L_a + L_b}{2}\\
  \implies 2L_a = L_a + L_b \implies L_a = L_b\\\\
  \lim_{n \to \infty} b_n = L_b = \sqrt{L_{a}L_b}\\
  \implies {L_b}^2 = L_{a}L_b \implies L_b = L_a
\end{gather*}

\newpage
\centerline{\bf Soluzione Esercizio 2 }
\bigskip

{\bf ii)}
\begin{gather*}
  \text{Applico il criterio del } \textbf{confronto asintotico} \text{ con la serie } \sum \frac{1}{e^n}:\\\\
  \lim_{n \to \infty} e^n \times {\left({\frac{n^2+1}{n^2+n+1}}\right)}^{n^2} = \lim_{n \to \infty} e^n \times {\left(\frac{\cancel{n^2}(1+\frac{1}{n^2})}{\cancel{n^2}(1+\frac{1}{n}+\frac{1}{n^2})}\right)}^{n^2} = \lim_{n \to \infty} e^n \times \frac{{(1+\frac{1}{n^2})}^{n^2}}{{(1+\frac{1}{n}+\frac{1}{n^2})}^{n^2}}\\
  = \lim_{n \to \infty} e^n \times \frac{e}{{(1+\frac{1}{n}+\frac{1}{n^2})}^{n^2}} = \lim_{n \to \infty} \frac{e^{n+1}}{{(1+\frac{1}{n}+\frac{1}{n^2})}^{n^2}}\\\\
  \text{Valutiamo il valore al limite del denominatore:}\\
  \lim_{n \to \infty}{\left(1+\frac{1}{n}+\frac{1}{n^2}\right)}^{n^2} = \lim_{n \to \infty} e^{\ln{\left(1+\frac{1}{n}+\frac{1}{n^2}\right)}{n^2}}\\
  \lim_{n \to \infty} \left(1+\frac{1}{n}+\frac{1}{n^2}\right) = 1 \implies \ln{\left(1+\frac{1}{n}+\frac{1}{n^2}\right)} \sim \left(1+\frac{1}{n}+\frac{1}{n^2}\right) - 1 = \frac{1}{n}+\frac{1}{n^2}\\
  \lim_{n \to \infty} {\left(\frac{1}{n}+\frac{1}{n^2}\right)n^2} = \lim_{n \to \infty} {\frac{n{\cancel{^2}}}{\cancel{n}}+\frac{\cancel{n^2}}{\cancel{n^2}}} = n + 1\\
  \implies {\left(1+\frac{1}{n}+\frac{1}{n^2}\right)}^{n^2} \sim \, e^{n + 1}\\\\
  \lim_{n \to \infty} \frac{e^{n+1}}{{\left(1+\frac{1}{n}+\frac{1}{n^2}\right)}^{n^2}} = \frac{e^{n+1}}{e^{n+1}} = 1\\\\
  \text{Siccome } \sum \frac{1}{e^n} \textbf{ converge} \quad \quad \text{(vedi punto } \textbf{(4)} \text{ in appendice)}\\
  \implies \sum {\left({\frac{n^2+1}{n^2+n+1}}\right)}^{n^2} \quad \textbf{converge}
\end{gather*}

\newpage
\centerline{\bf Soluzione Esercizio 2 }
\bigskip

{\bf i)}
\begin{gather*}
  \frac{\sqrt{n^2+1} - \sqrt[3]{n^3+1}}{1} \times \frac{n^2+1+{\sqrt[3]{n^3+1}}^2+\sqrt{n^2+1}\sqrt[3]{n^3+1}}{n^2+1+{\sqrt[3]{n^3+1}}^2+\sqrt{n^2+1}\sqrt[3]{n^3+1}} = \frac{n^2\sqrt{n^2+1} + \sqrt{n^2+1} - n^3 - 1}{n^2+1+{\sqrt[3]{n^3+1}}^2+\sqrt{n^2+1}\sqrt[3]{n^3+1}}\\\\
   = \frac{n^2\sqrt{n^2+1}-n^3}{n^2+1+{\sqrt[3]{n^3+1}}^2+\sqrt{n^2+1}\sqrt[3]{n^3+1}} + \frac{\sqrt{n^2+1}}{{n^2+1+{\sqrt[3]{n^3+1}}^2+\sqrt{n^2+1}\sqrt[3]{n^3+1}}}\\\\
  - \frac{1}{n^2+1+{\sqrt[3]{n^3+1}}^2+\sqrt{n^2+1}\sqrt[3]{n^3+1}}\\\\
  \text{Quindi avremo che }\\ \sum \sqrt{n^2+1} - \sqrt[3]{n^3+1} = \sum a_n + \sum b_n + \sum c_n\\\\
  \text{Non c'è il rischio di ottenere una } \textbf{forma indeterminata} \,\, \infty - \infty \text{, infatti:}\\
  c_n \sim \frac{1}{3n^2} \implies c_n \text{ è } \textbf{convergente}\\
  b_n \sim \frac{n}{3n^2} \sim \frac{1}{n} \implies b_n \text{ è } \textbf{positivamente divergente}\\\\
  a_n = \frac{n^2\sqrt{n^2+1}-n^3}{n^2+1+{\sqrt[3]{n^3+1}}^2+\sqrt{n^2+1}\sqrt[3]{n^3+1}} \times \frac{n^2\sqrt{n^2+1}+n^3}{n^2\sqrt{n^2+1}+n^3}\\\\
  = \frac{\cancel{n^6}+n^4-\cancel{n^6}}{(n^2+1+{\sqrt[3]{n^3+1}}^2+\sqrt{n^2+1}\sqrt[3]{n^3+1})(n^2\sqrt{n^2+1}+n^3)}\\\\
  = \frac{n^4}{(n^2+1+{\sqrt[3]{n^3+1}}^2+\sqrt{n^2+1}\sqrt[3]{n^3+1})(n^2\sqrt{n^2+1}+n^3)} \sim \frac{n^4}{2n^5} \sim \frac{1}{n}\\\\
  \implies \quad \sum a_n \text{ è } \textbf{positivamente divergente}\\\\\\
  \implies \sum \sqrt{n^2+1} - \sqrt[3]{n^3+1} \textbf{ diverge positivamente} \quad \text{(vedi punto } \textbf{(3)} \text{ in appendice)}
\end{gather*}

\newpage
\centerline{\bf Soluzione Esercizio 3 }
\bigskip

\begin{gather*}
  \text{Consideriamo il punto } m = \frac{b-a}{2}\\
  \text{Sviluppiamo la } \textbf{formula di Taylor} \text{ (con } \textbf{resto di Lagrange} \text{) centrata in } a \text{ e } b\\
  f(m) = f(a) + \cancel{f^{'}(a)(m-a)} + \frac{f^{''}(\psi)}{2}(m-a)^2\\
  f(m) = f(b) + \cancel{f^{'}(b)(m-b)} + \frac{f^{''}(\omega)}{2}(m-b)^2\\
  \text{Per qualche } \psi, \omega \in (a,b)\\\\
  f(b) + \frac{f^{''}(\omega)}{2}(m-b)^2 = f(a) + \frac{f^{''}(\psi)}{2}(m-a)^2\\\\
  f(b) - f(a) = \frac{f^{''}(\psi)}{2}\left(\frac{1}{2}(b-a)\right)^2 - \frac{f^{''}(\omega)}{2}\left(\frac{1}{2}(b-a)\right)^2 = \left[\frac{f^{''}(\psi)}{2} - \frac{f^{''}(\omega)}{2}\right]\frac{1}{4}(b-a)^2\\\\
  4\times\frac{[f(b) - f(a)]}{(b-a)^2} = \left[\frac{f^{''}(\psi)}{2} - \frac{f^{''}(\omega)}{2}\right] \quad \implies \quad 4\times\frac{|f(b) - f(a)|}{(b-a)^2} = \frac{|f^{''}(\psi) - f^{''}(\omega)|}{2}\\\\
  \text{Per la disuguaglianza } \textbf{(2)} \text{ in appendice:}\\
  \frac{|f^{''}(\psi) - f^{''}(\omega)|}{2} \leq \max{(|f^{''}(\psi)|,|f^{''}(\omega)|)} = |f^{''}(\theta)|\\\\
  4\times\frac{|f(b) - f(a)|}{(b-a)^2} \leq |f^{''}(\theta)|\\
  \text{Con } \theta = \psi \quad \text{ oppure } \quad \theta = \omega\\
\end{gather*}

\newpage
\centerline{\bf Soluzione Esercizio 4 }
\bigskip

{\bf i)}
\begin{gather*}
  \cos(x) = \cos\left(2 \times \frac{x}{2}\right) = 1-2\sin^2\left(\frac{x}{2}\right)\\\\
  x \neq 0 \implies \sin(x) < |x|\\
  \implies 1-2\sin^2\left(\frac{x}{2}\right) > 1-2\left(\frac{|x^2|}{4}\right) = 1-\frac{x^2}{2}\\
  \implies \cos(x) > 1-\frac{x^2}{2} \quad \text{per} \quad x\neq0\\
\end{gather*}

{\bf ii)}
\begin{gather*}
  \text{Se per } \textbf{assurdo} \quad \exists \, x>0: \quad x-\frac{x^3}{3!} \geq \sin(x)\\
  \implies \quad \frac{x-\sin(x)}{\frac{x^3}{3!}} \geq 1\\\\
  f(x) := x-\sin(x) \quad g(x) := \frac{x^3}{3!}\\
  f(0) = 0 \quad g(0) = 0\\\\
  \text{Per il teorema di } \textbf{Cauchy}\\
  \frac{f(x)-f(0)}{g(x)-g(0)} = \frac{f(x)}{g(x)} = \frac{f^{'}(\theta)}{g^{'}(\theta)} = \frac{1-\cos(\theta)}{\frac{\theta^2}{2}}\\\\
  \text{Applichiamo ancora lo stesso procedimento:}\\
  \frac{1-\cos(\theta)}{\frac{\theta^2}{2}} = \frac{\sin(\psi)}{\psi}\\\\
  \frac{x-\sin(x)}{\frac{x^3}{3!}} \geq 1 \iff \exists \, \psi \in (0,\theta): \quad \text{(con } 0< \theta <x \text{)}\\
  \frac{\sin(\psi)}{\psi} \geq 1\\\\
  \text{che è } \textbf{impossibile} \quad \text{(vedi punto } \textbf{(1)} \text{ in appendice)}
\end{gather*}

\newpage
\centerline{\bf Soluzione Esercizio 4 }
\bigskip

{\bf iii)}
\begin{gather*}
  \text{Se per } \textbf{assurdo} \quad \exists \, x>0: \quad x-\frac{x^3}{3!}+\frac{x^5}{5!} \leq \sin(x)\\
  \implies \quad \frac{\sin(x)+\frac{x^3}{3!}-x}{\frac{x^5}{5!}} \geq 1\\\\
  \text{Applichiamo il teorema di } \textbf{Cauchy} \text{ (con lo stesso procedimento del punto } \textbf{(2)} \text{)} \text{:}\\
  \frac{\sin(x)+\frac{x^3}{3!}-x}{\frac{x^5}{5!}} = \frac{\cos(\theta)+\frac{\theta^2}{2}-1}{\frac{\theta^4}{4!}} = \frac{-\sin(\psi)+\psi}{\frac{\psi^3}{3!}}\\
  = \frac{1-\cos(\upsilon)}{\frac{\upsilon^2}{2}} = \frac{\sin(\lambda)}{\lambda}\\\\
  \frac{\sin(x)+\frac{x^3}{3!}-x}{\frac{x^5}{5!}} \geq 1 \iff \exists \, \lambda \in (0,\upsilon): \quad \text{(con } \upsilon < \psi < \theta < x \text{)}\\
  \frac{\sin(\lambda)}{\lambda} \geq 1\\\\
  \text{che è } \textbf{impossibile} \quad \text{(vedi punto } \textbf{(1)} \text{ in appendice)}
\end{gather*}

\newpage
\centerline{\bf APPENDICE }
\bigskip

{\bf 1)}
\begin{gather*}
  \sin(x) < x \quad \text{per} \quad x>0\\\\\\
  \text{Se per } \textbf{assurdo} \quad \exists \, x>0: \quad \sin(x) > x\\
  \implies \frac{\sin(x)}{x} > 1\\\\
  f(x) := \sin(x) \quad g(x) := x\\
  f(0) = 0 \quad g(0) = 0\\\\
  \text{Per il teorema di } \textbf{Cauchy}\\
  \frac{f(x) - f(0)}{g(x) - g(0)} = \frac{f(x)}{g(x)} = \frac{f^{'}(x)}{g^{'}(x)}\\\\
  \exists \, \theta \in (0,x):\\
  \frac{\sin(x)}{x} = \cos(\theta) > 1 \quad \text{(} \textbf{assurdo} \text{)}\\
  -------------------------\\
  \sin(x) = x \quad \iff \quad x = 0\\\\\\
  f(x) := \sin(x) - x \quad f(0) = 0\\
  \text{Supponiamo per } \textbf{assurdo} \text{ che } \exists \, x_0 \in \, (0,1]: \quad \sin(x_0) = x_0\\\\
  \text{Applichiamo il } \textbf{teorema di Lagrange}:\\
  \frac{f(x_0) - f(0)}{x_0 - 0} = f^{'}(\theta) \quad \text{con } \theta \in (0,x_0)\\
  \frac{0}{x_0} = 0 = \cos(\theta) - 1 \quad \implies \quad \cos(\theta) = 1\\\\
  \nexists \, \theta \in (0,x_0): \quad \cos(\theta) = 1\\
  \implies \quad \nexists \, x_0 \in \, (0,1]: \quad \sin(x_0) = x_0
\end{gather*}

\newpage
\centerline{\bf APPENDICE }
\bigskip

{\bf 2)}
\begin{gather*}
  \frac{|x-y|}{2} \leq \max{(|x|,|y|)}\\\\\\
  \text{Per la disuguaglianza triangolare:}\\
  |x+(-y)| \leq |x| + |y|\\\\
  M := \max{(|x|,|y|)}\\
  |x| + |y| \leq 2M\\\\
  \implies \quad |x-y| \leq 2M\\
  \frac{|x-y|}{2} \leq \max{(|x|,|y|)}\\
\end{gather*}

\newpage
\centerline{\bf APPENDICE }
\bigskip

{\bf 3)}
\begin{gather*}
  \text{Se }\\
  \bullet \,\sum a_n = \sum b_n + \sum c_n + \sum d_n\\
  \bullet \quad \sum b_n, \, \sum c_n \text{ divergono positivamente}\\
  \bullet \quad \sum d_n \text{ converge}\\\\
  \implies \sum a_n \text{ diverge positivamente}\\\\\\
  \text{Verifichiamo la definizione per } \sum a_n:\\
  \text{Scegliamo un } M \text{ qualsiasi}\\
  L := \sum d_n (< +\infty)\\\\
  \text{Scegliamo } n_b:\\
  \forall \, n \geq n_b \quad \sum_{i=1}^{n} b_i \geq \frac{M}{2} + \frac{|L|}{2}\\\\
  \text{Analogamente scegliamo } n_c:\\
  \forall \, n \geq n_c \quad \sum_{i=1}^{n} c_i \geq \frac{M}{2} + \frac{|L|}{2}\\\\
  \alpha := \max{(n_b, n_c)}\\
  \implies \forall \, n \geq \alpha \quad \sum_{i=1}^{n} a_i \geq \frac{M}{2} + \frac{|L|}{2} + \frac{M}{2} + \frac{|L|}{2} + L = M + |L| + L \geq M\\\\
  \text{Quindi } \sum a_n \text{ è } \textbf{divergente}\\
  \text{Lo stesso ragionamento vale per un numero differente di serie}\\
  \text{divergenti positivamente o convergenti}
\end{gather*}

\newpage
\centerline{\bf APPENDICE }
\bigskip

{\bf 4)}
\begin{gather*}
  \sum \frac{1}{e^n} \quad \textbf{converge}\\\\\\
  \text{Dimostriamo per } \textbf{induzione} \text{ che }
  e^n > n^2 \quad \forall \, n\\
  1) \quad e > 1\\
  n) \quad e^{n+1} = e^n \times e = (e-1)e^n + e^n\\
  (n+1)^2 = n^2 + 2n + 1 < e^n + 2n + 1 < e^n + (e-1)e^n\\\\
  \text{Dimostriamo per } \textbf{induzione} \text{ che }
  (e-1)e^n > 2n+1 \quad \forall \, n\\
  1) \quad 4.67 > 3\\
  n) \quad (e-1)e^{n+1} = (e-1)e^n \times e = (e-1)(e-1)e^n + (e-1)e^n = (e-1)^2e^n + (e-1)e^n\\
  2(n+1)+1 = 2n+3 < (e-1)e^n+2 < (e-1)e^n + (e-1)^2e^n\\\\
  \text{Dimostriamo che } (e-1)^2e^n > 2 \quad \forall \, n\\
  1) \quad 8.02 > 2\\
  n) \quad (e-1)^2e^n \text{ è } \textbf{crescente}\\\\
  e^n > n^2 \quad \forall \, n\\
  \implies \frac{1}{e^n} < \frac{1}{n^2} \quad \forall \, n\\\\
  \text{Quindi } \sum \frac{1}{e^n} \textbf{ converge} \text{ per il criterio del } \textbf{confronto}
\end{gather*}

\end{document}
