\documentclass{article}
\usepackage{amsmath}
\usepackage[left=2cm, right=2cm, top=2cm,bottom = 2cm]{geometry}
\begin{document}

\begin{titlepage}
   \vspace*{\stretch{1.0}}
   \begin{center}
      \Large\textbf{Prova bonus Analisi 1}\\
      \large{Francesco Andreuzzi}\\
      \large{12 ottobre 2018}
   \end{center}
   \vspace*{\stretch{2.0}}
\end{titlepage}

\begin{gather*}
\textbf{Esercizio 1.}\\*\\*\\*
\text{Dimostriamo la formula per la somma dei numeri da n a 2n:}\\*
n + .. + 2n = (n+1)*\frac{3}{2}*n\\*
\text{Per n = 1}\\*
S_1 = 3\\*
\text{Considerando valido il caso n-esimo, si può dimostrare il caso n+1-esimo}\\*
S_{n+1} - S_n = (2n+1) + (2n+2) - n\\*
S_{n+1} = ((n + 1) + 1)*\frac{3}{2}*(n+1) = (n+1)*\frac{3}{2}*n + (2n+1) + (2n+2) - (n)\\*
(n+2)*\frac{3}{2}*(n+1) = (n+1)*\frac{3}{2}*n + 3n+3\\*
(n+2)*\frac{3}{2}*(n+1) = (n+1)*\frac{3}{2}*n + 3(n+1)\\*
(n+2)*\frac{3}{2}*(n+1) = (n+1)*(\frac{3}{2}*n + 3)\\*
(n+2)*\frac{3}{2}*(n+1) = (n+1)*3(\frac{1}{2}*n + 1)\\*
(n+2)*\frac{3}{2}*(n+1) = (n+1)*3(\frac{n+2}{2})\\*
(n+2)*\frac{3}{2}*(n+1) = (n+1)*\frac{3}{2}(n+2)\\*
\text{Quindi la formula è valida per n} \geq 1\\*\\*
A_m \geq H_m\\*
\frac{\frac{1}{n} + ... + \frac{1}{2n}}{n+1} \geq \frac{n+1}{n + ... + 2n}\\*
\frac{\frac{1}{n} + ... + \frac{1}{2n}}{n+1} \geq \frac{n+1}{(n+1)*\frac{3}{2}*n}\\*
\frac{\frac{1}{n} + ... + \frac{1}{2n}}{n+1} \geq \frac{2*(n+1)}{(n+1)*3*n}\\*
\frac{\frac{1}{n} + ... + \frac{1}{2n}}{n+1} \geq \frac{2}{3*n}\\*
\frac{1}{n} + ... + \frac{1}{2n} \geq \frac{2*(n+1)}{3*n}\\*
\frac{1}{n} + ... + \frac{1}{2n} \geq \frac{2}{3} * \frac{n+1}{n}\\*
\frac{1}{n} + ... + \frac{1}{2n} \geq \frac{2}{3} * (1 + \frac{1}{n}) >  \frac{2}{3}\\*
\frac{1}{n} + ... + \frac{1}{2n} >  \frac{2}{3}
\end{gather*}

\pagebreak

\begin{gather*}
\textbf{Esercizio 2.}\\*\\*\\*
\text{Dimostriamo la formula per la somma dei numeri da n+1 a 3n+1:}\\*
(n+1) + ... + (3n+1) = (2n+1)^2\\*
\text{Per n = 1 si ottiene}\\*
S_1 = 9\\*
\text{Supponiamo che valga il caso (n), e cerchiamo di dimostrare il caso (n+1)}\\*
S_{n+1} - S_n = (3n+4) + (3n+3) + (3n+2) - (n+1)\\*
S_{n+1} = [2(n+1) + 1]^2 = (2n+1)^2 + (8n+8)\\*
(2n+3)^2 = (2n+1)^2 + 8(n+1)\\*
4n^2 + 12n + 9 = 4n^2 + 4n +1 + 8n + 8\\*
4n^2 + 12n + 9 = 4n^2 + 12n + 9\\*
\text{Quindi la formula è valida per n} \geq 1\\*\\*
A_m \geq H_m\\*
\frac{\frac{1}{n+1} + ... + \frac{1}{3n+1}}{2n+1} \geq \frac{2n+1}{(2n+1)^2}\\*
\frac{1}{n+1} + ... + \frac{1}{3n+1} \geq \frac{(2n+1)(2n+1)}{(2n+1)^2}\\*
\frac{1}{n+1} + ... + \frac{1}{3n+1} \geq \frac{(2n+1)^2}{(2n+1)^2}\\*
\frac{1}{n+1} + ... + \frac{1}{3n+1} \geq 1\\*\\*
\text{Dimostriamo che non vale } S_n = 1\\*
\text{1) } S_1 = \frac{13}{12} > 1\\*
\text{Dobbiamo verificare che partendo da un } S_1 > 1 \text{ è impossibile arrivare ad un } S_n = 1\\*
\text{Ma } S_n \text{ è} \textbf{ crescente} \text{. Infatti si ha che:}\\*
 S_{n+1} - S_n = \frac{1}{3n+2} + \frac{1}{3n+3} + \frac{1}{3n+4} - \frac{1}{n+1}\\*
\text{Poniamo } a = \frac{1}{3n+2} + \frac{1}{3n+3} + \frac{1}{3n+4}\\*
\text{Poniamo } b = \frac{1}{n+1}\\*
\text{Si può verificare che } a > b \text{ e quindi:}\\*
\frac{1}{3n+2} + \frac{1}{3n+3} + \frac{1}{3n+4} > \frac{1}{n+1}\\*\\*
\Rightarrow S_n > S_1 > 1 \text{ } \forall n
\end{gather*}

\pagebreak

\begin{gather*}
\textbf{Esercizio 3.}\\*\\*\\*
\text{Dimostriamo la formula per la somma dei numeri da 3n+1 a 5n+1:}\\*
(3n+1) + ... + (5n+1) = (2n+1)(4n+1)\\*
\text{Per n = 1 si ottiene}\\*
S_1 = 15\\*
\text{Supponiamo che valga il caso (n), e cerchiamo di dimostrare il caso (n+1)}\\*
S_{n+1} - S_n = (5n + 6) + (5n + 5) + (5n + 4) + (5n + 3) + (5n + 2) - (3n+1) - (3n+2) - (3n+3)\\*
S_{n+1} =[2(n+1) + 1][4(n+1) + 1] = (2n+1)(4n+1) + (16n+14)\\*
(2n+3)(4n+5) = 8n^2+2n+4n+1 + 16n+14\\*
8n^2+10n+12n+15 = 8n^2+22n+15\\*
8n^2+22n+15 = 8n^2+22n+15\\*
\text{Quindi la formula è valida per n} \geq 1\\*\\*
A_m \geq H_m\\*
\frac{\frac{1}{3n+1} + ... + \frac{1}{5n+1}}{2n+1} \geq \frac{2n+1}{(2n+1)(4n+1)}\\*
\frac{\frac{1}{3n+1} + ... + \frac{1}{5n+1}}{2n+1} \geq \frac{1}{4n+1}\\*
\frac{1}{3n+1} + ... + \frac{1}{5n+1} \geq \frac{2n+1}{4n+1} > \frac{1}{2} \text{    infatti } (\frac{4n+2}{4n+1} > 1)\\*
\Rightarrow \frac{1}{3n+1} + ... + \frac{1}{5n+1} > \frac{1}{2}\\*\\*
\text{Dimostriamo che } \frac{1}{3n+1} + ... + \frac{1}{5n+1} < \frac{2}{3}\\*
S_1 = \frac{1}{4} + \frac{1}{5} + \frac{1}{6} = \frac{15+12+10}{60} = \frac{37}{60} < \frac{2}{3}\\*
\text{Consideriamo a e b tali che}\\*
a = \frac{1}{5n+6} + \frac{1}{5n+5} + \frac{1}{5n+4} + \frac{1}{5n+3} + \frac{1}{5n+2}\\*
b = \frac{1}{3n+1} + \frac{1}{3n+2} + \frac{1}{3n+3}\\*\\*
S_{n+1} = S_n - b + a\\*\\*
\text{Si dimostra facilmente che}\\*
a < b\\*
\frac{1}{5n+6} + \frac{1}{5n+5} + \frac{1}{5n+4} + \frac{1}{5n+3} + \frac{1}{5n+2} < \frac{1}{3n+1} + \frac{1}{3n+2} + \frac{1}{3n+3}\\*
\text{E quindi la quantità che viene } \textbf{aggiunta} \text{ al variare di n ad } S_n \text{ è minore di quella che viene } \textbf{sottratta.}\\*
\text{Da questa informazione ricaviamo che } S_n > S_{n+1} \text{ (e quindi } S_n \text{ è } \textbf{decrescente} \text{).}\\*
\Rightarrow S_n < S_1 < \frac{2}{3} \text{ } \forall n\\*
\end{gather*}

\end{document}